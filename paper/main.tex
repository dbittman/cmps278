\documentclass[twocolumn,10pt]{article}
\usepackage[margin=1in]{geometry}
\usepackage[T1]{fontenc}
\usepackage[sort]{natbib}
\usepackage{titling}
\usepackage{titlesec}
\usepackage{graphicx}
\usepackage{siunitx}
\usepackage{amsmath,xspace}
\usepackage[backgroundcolor=yellow]{todonotes}
\usepackage[utf8]{inputenc}
\usepackage{bibentry}
\usepackage{float}
\usepackage[ttscale=.875]{libertine}
\usepackage{listings}
\usepackage{libertinust1math}
\usepackage{mdframed} %nice frames
\usepackage[format=plain,
            labelfont={bf,it},
            textfont=it]{caption}
\usepackage{listings}
\usepackage[hyphens]{url}
\usepackage[breaklinks]{hyperref}





\author{Daniel Bittman \\ dbittman@ucsc.edu}
\title{NVKV: Building a KV Store for Non-Volatile Memory\\{\small CMPS278:
Database Design and Implementation: Class Project\\\vspace{-5mm}Instructor: Shel Finkelstein}}

\newcommand{\etal}{\emph{et~al.}\xspace}
\newcommand{\bdb}{Berkeley DB\xspace}
\begin{document}
\biolinum
\maketitle
\libertine 
\renewcommand\ttdefault{lmtt}

\lstset{ 
  backgroundcolor=\color{white},   % choose the background color; you must add \usepackage{color} or \usepackage{xcolor}; should come as last argument
  basicstyle=\ttfamily\footnotesize,        % the size of the fonts that are used for the code
  breakatwhitespace=false,         % sets if automatic breaks should only happen at whitespace
  breaklines=true,                 % sets automatic line breaking
  captionpos=b,                    % sets the caption-position to bottom
  deletekeywords={...},            % if you want to delete keywords from the given language
  escapeinside={\%*}{*)},          % if you want to add LaTeX within your code
  extendedchars=true,              % lets you use non-ASCII characters; for 8-bits encodings only, does not work with UTF-8
  keepspaces=true,                 % keeps spaces in text, useful for keeping indentation of code (possibly needs columns=flexible)
}





\section*{Abstract}


%TO cite:
%kolli:asplos16 lu:tos16, wang:cal15(transactionS)


\section{Introduction}

Byte-addressable non-volatile memory (BNVM) promises to fundamentally change
how applications access persistent
storage~\cite{lee_architecting_2009,fox:2001feram,sttram,wong2010phase,intel3dxpoint},
especially when the memory hierarchy changes to include BNVM along-side DRAM.
Changes are expected across the system stack~\cite{condit:sosp09}, from
processors introducing new feature sets, to operating systems providing new I/O
models, to applications being redesigned around low-latency persistent memory. Of
particular note is the additional \textit{power} given to middleware
applications---the operating system will provide more direct access to
persistent storage to middleware applications and
libraries~\cite{bittman-ssrctr-17-01}, allowing these applications more control
over their storage techniques and access without operating system interposition.
This has been a long debated topic, and with the advent of BNVM the debate is
coming closer to being settled in favor of applications getting more control and
power.

Of course, with greater power comes greater responsibility, and this is even
more true with BNVM. Data consistency issues threaten most applications, and the
problems of consistency are only magnified with the fine-grained writes of
byte-addressable storage. Whereas writes to DRAM would disappear when power is
cycled, corrupted data structures in BNVM persist across power cycles, meaning
applications must expend additional effort to prevent such corruption that was
not required when the storage hierarchy was separated\footnote{We could
rely on block-oriented storage to ease this.}. An additional
difficulty is energy use, where prior systems could both buffer writes in DRAM
(which had power usage characteristics that were largely independent of write
bandwidth) and coalesce them to persistent storage. With BNVM, however, the
power scales with write bandwidth at a much higher rate, meaning writes must be
minimized to minimize power. Minimizing writes has the additional benefit of
reducing wear on the memory cells, which is important since many candidate
technologies have significantly lower write endurance than DRAM.

To explore these problems, we have developed NVKV, a non-volatile key-value
store that provides backwards compatibility with the \bdb programming interface.
The prototype implements insert and lookup, although delete would also be trivial
to implement. It organizes and stores data in such a way so that it is easy to
perform all operations using simple and small memory transactions provided by
future software or hardware transactional memory. To support such a design, it
indexes data using a variant of Cuckoo hashing that does not require moving
items when rehashing them. We found that this design allowed the transactions
required to implement insert to be small (fewer than 5 memory access), and that
it was relatively easy to ensure the database could never be corrupted by
unexpected power failures. Furthermore, we found the performance to be
significantly improved over \bdb, likely because our system was designed for
such a memory hierarchy, where \bdb was not. Finally, the Cuckoo hashing variant
we designed resulted in significantly reduced data movement, which is important
for reducing power and wear on BNVM.

The main contributions of this work are:

\begin{enumerate}

\item We motivate the need to reevaluate the design of a key-value store for
BVNM.
\item We build a key-value store, NVKV, which uses transactional
memory to implement updates, and show that it has significantly improved
performance over \bdb.
\item We show that a simple mechanism to minimize writes (simply not reinserting
after expanding the hash table) is viable and does not significantly hurt
overall performance (and in fact smooths latency spikes caused by expanding the
hash table).
\end{enumerate}



\section{Background}

Non-volatile memory technologies, including phase change memory
(PCM)~\cite{lee_architecting_2009,wong2010phase}, Ferroelectric RAM
(FeRAM)~\cite{fox:2001feram}, and spin-torque transfer RAM (STT-RAM)~\cite{sttram},
among others, promise to fundamentally change the design of our
devices, operating systems, and applications. Although the technologies are
starting to make their way into consumer devices~\cite{intel3dxpoint}, their
full potential will be seen when they replace or exist alongside DRAM on the
memory bus (as byte-addressable non-volatile memory, BNVM).
Such a memory hierarchy will allow the processor, and therefore
applications, to access persistent storage with normal load and store
instructions, bypassing the high-latency I/O operations of the operating system.

The advent of these technologies means that we must build new applications
explicitly designed for them. Whereas existing key-value stores would work when
given access to a file in a file system on BNVM, the result would be sub-optimal
because it would still have the operating system interposing on access to
persistent storage, which is unnecessary for BNVM. Furthermore, the access would
be block oriented, which is irrelevant for these technologies. Finally, the
consistency mechanisms enforced by file systems would be overkill compared to
the more light-weight consistency that could be achieved by a key-value store
designed for BNVM. Together, these all act as additional overhead for key-value
stores that could be avoided by properly redesigning and reevaluating the needs
of such a system. That is a goal of NVKV; understanding the underlying
technology and designing for it.

While this is not new, the particular system model NVKV is built for is. We are
entering an era where we may see a range of persistent memory on the memory bus
from no persistent memory to \textit{only} persistent memory. While some exising
key-value stores are built for persistent memory, they often assume the
existence of DRAM~\cite{echo,Arulraj:2016wbl}. However, IoT devices may soon see
their use of DRAM diminish as more power-friendly technologies come
out~\cite{Jayakumar2014powering}. In fact, from a survey on
papers~\cite{dhiman_pdram:_2009,lee_architecting_2009,xiangyu_dong_nvsim:_2012,qureshi_scalable_2009,Chen_rethinkingdatabase,bedeschi_8mb_2004} discussing
energy consumption of PCM and DRAM, we can see that the energy cost per bit to
write to PCM is 50 times more expensive, but
the idle cost of DRAM is a \textit{billion} times the write cost per bit. This
means that the idle cost of DRAM can only be surpassed by PCM when writing a
billion bits per second. While this is large, it is not unreachable---systems
must be careful to avoid excessive copies and rewrites when using PCM, something 
existing system do not always optimize for. Finally, since PCM is
extremely energy efficient in read-mostly workloads, it will likely find its
place in IoT device without DRAM. Thus, key-value stores must be designed for
this all-BNVM model.

\paragraph{Transactional Memory}

Many BNVM systems focus on consistency of data structures in due to the
persistent nature of BNVM. When applications store persistent data in
persistent memory, they expect the data to remain consistent with respect to a
set of invariants (e.g. a linked list is valid, etc). However, with the fine
granularity of writes to BNVM and the ability for power failures to strike at
any time, ensuring consistency on BNVM is an important challenge that must be
addressed in new ways.

There are many ways to provide applications with methods for ensuring
consistency, ranging from persisting during shutdown with
battery-backups~\cite{narayanan:asplos12} to write-through caching and careful
ordering with atomics~\cite{bhandari2012implications}, to explicit flushing and
fencing~\cite{condit:sosp09}. While these mechanisms are usable, they range from
being non-portable to being incredibly inefficient. Only explicit flushing is
close to reasonable in terms of both usability and performance, but it has the
same problems as atomic memory accesses without programming language support.
Only with proper language support can persistent memory be both high performance
and easily usable.

The exact form of this language support is still debated, partly because it will
depend on the type of hardware support we see for persistent memory programming.
Hardware support could range from persist barriers to support for transactional
memory. However, existing hardware transactional memory (e.g. in Intel
processors) is deprecated due to bugs. Instead, we are left with enough support
for software transactional memory, which has similar problems as other
approches, namely performance and usability~\cite{stm}. However, hardware
transactional memory \textit{could} be a
reality~\cite{kolli:asplos16,lu:tos16,wang:cal15}---a reality this paper
assumes. Thus, we are focusing on developing a key-value store built for BNVM
with hardware (or software) transactional memory support to ensure consistency.
As an optimization, we additionally make use of weaker consistency mechanisms,
such as fencing and flushing, when transactions would be too heavy-weight for
the updates we are protecting.



\section{Design}

The design goals of NVKV are five-fold:
\begin{enumerate}
	\item \textbf{Direct Access}: BNVM should be operated on \textit{directly},
		without operating system interposition and without additional copies, as
		much as possible. This is possible in Linux using BNVM drivers and DAX
		\texttt{mmap}~\cite{dax}, wherein an application can memory map the
		actual BNVM directly into its address space.
	\item \textbf{Consistency with Operations}: Once each operation returns to
		the application, any changes required by the operation must be
		persistent. For example, after the insert function returns, the new data
		and the updated information in the index must be persisted to BNVM and
		be visible to the applications when it looks for it.
	\item \textbf{Consistency with Power Failures}: NVKV must support power
		failures at \textit{any} point during \textit{any} operation. When power
		is resumed, and the application restarted, it should be able to continue
		operating without the database having been corrupted. In this case, we
		define free from corruption as all values that were inserted can be
		found, all items that were deleted cannot be found, all data that was in
		the database remains the same across power failures, and keys and values
		match as expected.

		If a power failure occurs in the middle of an insert, there is a
		possibility that the insert only partially completed. That is fine, as
		long as the above requirements are met. For example, the data might be
		copied in before the index is updated. If the power fails here, the only
		``inconsistency'' would be that there is a key-value pair that is not
		retrievable, which does not violate the above requirements because the
		insert operation had not yet returned to the user. However, this
		database state is sub-optimal since it contains unnecessary data. For
		these kinds up inconsistencies, we allow a fsck tool to be run to
		measure reachability for all keys and values and ensure that the
		database count is correct. Note that if the database were used without
		running the fsck tool, it would work as expected; the fsck tool is just
		there to optimize the database when power failures occur.
	\item \textbf{Performance}: Like any good key-value store, NVKV aims to
		provide the design requirements while being high performance.
		Specifically, it aims to be \textit{low-latency} to match the
		low-latency access to persistent storage that BNVM provides.
	\item \textbf{Compatibility}: To ease evaluation, NVKV provides an identical
		interface to Berkeley DB. When developing a new system, adoption is key,
		and backwards compatibility is vital for increasing adoption.
\end{enumerate}

The design choices in NVKV fit some common themes. The first theme is to make
each update to the database that is done by an operation like insert a series of
small, consistent sub-operations. Each of these sub-operations moves the
database from one valid state to another, possibly with the use of a transaction
(which, to avoid confusion, in this document means ``a transaction using
transactional memory on some memory''). The database is carefully designed to
allow it to \textit{always} construct each operation out of a bounded number of
sub-operations that keep the database in a consistent state. The second common
theme is to avoid data movement as much as possible. While this is not always
possible when rehashing, NVKV bounds the number of moved words on \textit{every}
insert.

One significant reason we strive for these goals is that we are designing a
system using a transactional memory primitive that does not exist yet. However,
we can expect that such a system will operate somewhat like a transaction: a
\texttt{transaction-begin}, followed by a bounded number of instructions,
followed by \texttt{transaction-commit} or \texttt{abort}. We expect the number
of instructions and the number of memory accesses allowed inside the transaction
to be bounded, at least due to the performance costs of having large
transactions, but at most because the hardware support might directly limit it.
Thus, we are designing pessimistically; should transactional memory support
exceed our expectations, NVKV will still work, however we can reasonably assume
that it will fit within the requirements of most reasonable transactional memory
proposals. The assumption is that both number of instructions and number of
memory accesses matter, although the number of memory accesses is more
important.

\subsection{Interface}

The interface to applications provided by NVKV is just like Berkeley DB. In
fact, the only difference is in the header file that must be included by the
application (\texttt{db.h} in the case of \bdb and \texttt{nvkv.h} in the case
of NVKV). Applications then link with \texttt{libnvkv.so}. The interfaces are
exactly the same, although a lot of functionality is left unimplemented, as full
compatibility is well outside the scope of a class project. The implemented
functionality is \texttt{db\_create}, \texttt{db\_open}, \texttt{get},
\texttt{put}, \texttt{sync}, \texttt{err}, \texttt{errx}, and manipulation of
\texttt{DBT}s\footnote{A \texttt{DBT} is a struct containing a pointer to memory
and a length, and is used to pass keys and values between the database library
and the application.}. The format of the database files is \textit{not} compatible; only
the programming interface is.

\subsection{Data Organization}

The database file contains two separate types of information: data and metadata. Data
refers to the contents of a \texttt{DBT} (keys and values) along with their
lengths. The exact format of the data storage is discussed in
section~\ref{sec:ds}. Metadata is the information kept by the database in order
to support the lookup operations on the data; that is, a small structure
containing information about the database as a whole, and the indexing data
structure.

\begin{figure}
\centering
\hspace*{-0.1in}
\includegraphics[width=84mm]{fig/addrspace}
\caption{Mapping the data and metadata address spaces to the physical address
space of the database file.}
\label{fig:addrspace}
\end{figure}

Since NVKV maintains backwards compatibility with \bdb, it is allowed only one file to
store both metadata and data. However, both of these grow when inserting,
meaning that if they were placed naively inside the database file, they might
have to be moved. Moving large amounts of data is fine in a database system
such as \bdb, but when operating directly on BNVM, we must avoid moving data
whenever possible, both to save energy, and because moving data is much more
difficult to do in a single transaction.

To avoid moving data, NVKV lays out the database file memory as shown in
figure~\ref{fig:addrspace}, where each page of metadata is interlaced with each
data page. Logically, this provides two separate, linear address spaces for data
and metadata that can both grow without interfering with each other. Internal pointers are stored \textit{as if} these address spaces
were real and started at zero; that is, if page-size is 4K and we wished to
write a pointer to the second page of metadata, it would be \texttt{0x1000 +
page-offset}, even though the actual location in the file this refers to is
\texttt{0x2000 + page-offset}. Storing pointers in such a format is done because
the database may not be mapped to the same location the next time it is loaded,
so we must translate pointers into a universal format when storing them.
When swizzling pointers into their virtual
address space value after \texttt{mmap}ing the database, NVKV relies on context
to choose the address space, apply the appropriate operation to determine the
offset into the file, and then add the base address of the \texttt{mmap}
region to calculate the ultimate virtual address.

The need to deal with persistent pointers in virtual memory is a real problem,
because we can no longer have an explicit serialization step when persisting to
persistent memory---after all, the data is already persistent. Thus, we must
always write pointers to memory in a universal format that applications can make
use of regardless of the ultimately mapped location of the data that is pointed
to or the pointers themselves. We are working on a larger project named
Twizzler~\cite{bittman-ssrctr-17-01} that addresses and explores these issues in
detail and provides a programming model based around transparent use of
universal pointers. If that system was fully functional, most of the complexity in this
key-value store would be trivial, since most of the code complexity comes from
dealing with swizzling between these different address spaces (data, metadata,
persistent, and virtual).


\subsection{Indexing}

The indexing structure we are using is a hash table, exploiting the
random-access and non-block-oriented nature of BNVM~\cite{Debnath:2016ht}. To
deal with collisions, we have chosen Cuckoo hashing~\cite{Pagh:2004}, a hashing
scheme that results in good cache locality, bounded lookups, and reasonably high
load-factor. Each key is hashed to two buckets, $b_1 = H_1(k) \mod l$ and $b_2 =
H_2(k) \mod l$, where $k$ is the key, $l$ is the table length, and $H_1$ and
$H_2$ are two different hash functions. When inserting, bucket $b_1$ is
evaluated, and if it is empty, the key is inserted there. If it is not empty, we
look at $b_2$. If \textit{it} is empty, we can insert there. Otherwise, we must
\textit{move} the contents of either $b_1$ or $b_2$ before we can insert. This
process is shown in figure~\ref{fig:insert}, where $B$ must move for $A$ to be
inserted.

\begin{figure}
\centering
\includegraphics[width=70mm,height=50mm]{fig/cuckoo_insert}
\caption{Inserting an element that requires moving an existing element to a
partner bucket.}
\label{fig:insert}
\end{figure}


To move $B$, we determine its partner bucket (the other bucket it hashes to),
and try to move it there. If there is an item in the partner bucket, we
recursively try to move until we are successful. Note that cycles are possible;
should a cycle occur (or a threshold number of moves is reached), the move is
aborted. When inserting, if both $b_1$ and $b_2$ are
full, and neither can be moved, the hash table is rehashed---the length is
doubled.

Lookup is implemented by checking both $b_1$ and $b_2$ for the key. This is a
strength of Cuckoo hashing, that lookup operations take a bounded, small number
of steps while still supporting reasonably efficient collision resolution.

\paragraph{Rehashing}

Rehashing is the most arduous part of dealing with hash tables, for it often
involves a large pause and a significant amount of data movement. Reinserting
all the keys into a double-length table is not viable for NVKV, because we are
trying desperately to avoid excessive data movement and because it is difficult
to define this in terms of a bounded transaction. A possible solution would be
to pass off the reinsertions into future insert operations, but that can get
complicated should a rehash be triggered a second time shortly thereafter.
Furthermore, this increases the complexity of insert operations.

Instead, NVKV takes the approach of just not reinserting the data. The table
length is still doubled, causing new inserts to be distributed throughout the
new table, but old data still remains clustered in the first half, untouched, as
shown in figure~\ref{fig:rehash}.
The lookup operation is modified to try looking up for each size the hash table
has been; that is, it calculates $b_1 = H_1(k) \mod l$ (and $b_2$ similarly),
followed by $b_1 = H_1(k) \mod \frac{l}{2}$, and so on, until it reaches the
original size of the hash table. We are essentially treating the table as a set
of tables, each twice the size of the previous, overlapping in memory.

Inserts are always done ``at the top level'', meaning that inserts and moves are always
done with the current value of $l$ and no other. When a collision is found, it
is moved to its partner bucket (calculated with $l$) even if it was originally
there from a previous value of $l$. Although the data from a previous $l$ is
clustered in the first half of the table ($l$ was just doubled; the second half
is empty), it is unlikely to cause a large number of movements immediately
because new inserts have equal likelihood to end up in the second half of the
table. Next, when an old value must move, it has equal likelihood of being moved
to the second half of the table, thereby eventually spreading the data out
across the table. The advantage of this is no extra code complexity; the data is
eventually spread through the normal operation of inserts and moves.


\begin{figure}
\centering
\hspace*{1mm}
\includegraphics[width=90mm]{fig/cuckoo_rehash}
\caption{Extending the hash table to level $n$ without reinserting existing
elements. Level $n-1$ has existing elements that are not moved during the
expansion, and the newly inserted element is more than twice as likely to find
an empty bucket to insert into.}
\label{fig:rehash}
\end{figure}

The end result is a hash table that does not require expensive rehashing while
providing bounded insert operations at the expense of slightly slower lookup
operations. However, we found that although get operations have to try the hash
table look up using several different values of $l$, in practice the data was redistributed
quite quickly, causing most insert operations to only lookup one level.


\paragraph{Why Cuckoo?}

Cuckoo hashing dove-tails nicely with the rehashing scheme. One of the reasons
this scheme would not work with something like linear probing is that lookups in
linear probing are not bounded. Looking up at each ``level'' (value of $l$) must
be bounded and fast for the additional requirement to lookup at each level to be
reasonable. Linear probing would end up searching a large number of buckets for
each lookup, which is infeasible, whereas Cuckoo hashing limits the bucket search at
each level to two.
Another advantage is that Cuckoo hashing already does data movement as part of
its implementation, meaning that the clustered data after a rehash will be
automatically spread out. When designing a system using transactions and complex
persistence requirements, reducing code complexity is a massive advantage.


For BNVM and transactional memory, we must both limit the size of the
sub-operations and make each one leave the index in a consistent state. Movement
in Cuckoo hashing achieves both goals; each movement requires only a small
number of memory operations onces the system decides to move a bucket somewhere.
The bucket movements also always move data to valid locations; they are never
temporarily in places they cannot be found.
If a key were to move from $b_1$ to $b_2$, it
will still be locatable because it \textit{could have} been in $b_2$ all along.
The lookup will still find it. This properly allows the recursive move function
to move data in these sub-operations, always with the index consistent. Even if
the recursion gets interrupted part-way through and power is lost, the database
will remain valid.




\subsection{Data Storage}
\label{sec:ds}

Data storage of keys and values is currently implemented much like an arena
allocator; an ``end'' pointer is kept and incremented whenever a new \texttt{DBT} is
recorded. Recording a \texttt{DBT} involves copying the data into the data
address space along with its length. The address of the copied-in length and
data is calculated, after which it can be used to refer to the data internally.
The end pointer is incremented by the data length plus the size of the length
field, allowing the next \texttt{DBT} to be written directly after it. Note
that, because of the address space interleaving described above, adding data to
the database never causes it to interfere with the metadata.

To ensure persistence and consistency, the end pointer is not updated until
after the data is copied in. After each word of data copied, the cache-line
is flushed and a memory fence is applied, ensuring that the data be flushed back
to BNVM.
The reason we do not use a transaction here is that the
data could be arbitrarily long. Once the data is copied, the end pointer is
incremented and flushed, again without a transaction because we are
only updating a single value.






\section{Implementation}

cuckoo:
order of lookup REALLY matters: reverse order.




cuckoo: each move goes from valid state to valid state with bounded
ops. Worst case the recursive move doesn't complete: still valid.

TXN may be too much overhead. instead, we can use persist-release. We do this
with count, because if the count is off a bit, that's okay. The only time that
happens is power fail, and we can recalculate the count after a fail if we need
it to be perfect.



Calc size of TXNs (num instruction, num of mem access). Also use Os to reduce just
those functions.



\begin{lstlisting}[caption={Transaction code, do\_move, optimized for speed.
Five instructions, five memory accesses (four writes).},label=lst:moveO3]
     vmovdqu (%rbx), %xmm0
     mov     %r9, 0x10(%r12)
     vmovups %xmm0, (%r12)
     movq    $0x0, 0x10(%rbx)
     movq    $0x0, (%rbx)
\end{lstlisting}

\begin{lstlisting}[caption={Transaction code, do\_insert, optimized for speed.
Four instructions, four memory accesses (three writes).},label=lst:insertO3]
    mov    -0x68(%rbp),%rdi
    mov    %r13,(%rax)
    movq   $0x1,0x10(%rax)
    mov    %rdi,0x8(%rax)
\end{lstlisting}








\begin{lstlisting}[caption={Transaction code, do\_move, optimized for size.
Eight instructions, seven memory accesses (five writes).},label=lst:moveOs]
     mov    (%rsi),%rax
     movslq %edx,%rdx
     mov    %rax,(%rdi)
     mov    0x8(%rsi),%rax
     mov    %rdx,0x10(%rdi)
     mov    %rax,0x8(%rdi)
     movq   $0x0,0x10(%rsi)
     movq   $0x0,(%rsi)
\end{lstlisting}

\begin{lstlisting}[caption={Transaction code, do\_insert, optimized for size.
Four instructions, three memory accesses (three writes).},label=lst:insertOs]
     movslq %r8d,%r8
     mov    %rdx,(%rsi)
     mov    %rcx,0x8(%rsi)
     mov    %r8,0x10(%rsi)
\end{lstlisting}





\section{Results and Analysis}


\subsection{Data Movement in the Hash Table}

\begin{figure}
\centering
\includegraphics[width=100mm]{fig/moves}
\caption{Stuff}
\label{fig:moves}
\end{figure}

\subsection{Transaction Sizes}

\subsection{Performance}

\begin{figure}
\centering
\hspace*{-0.5in}
\includegraphics[width=98mm]{fig/perf}
\caption{Stuff}
\label{fig:perf}
\end{figure}

\begin{figure}
\centering
\hspace*{-0.5in}
\includegraphics[width=98mm]{fig/line}
\caption{Stuff}
\label{fig:line}
\end{figure}




\section{Related Work}

\paragraph{BNVM Systems}

There has been a lot of recent work on providing clean and safe abstractions for
non-volatile memory, including data
structures~\cite{coburn:asplos11,hu:atc17,Yang:2015,Venkataraman:2011} and programming
models~\cite{ren:micro15,volos:asplos11,condit:sosp09,guerra:atc12,Narayanan:2012}.
Most of these provide a lower-level interface for applications
to be built atop non-volatile memory and focus on consistency and durability. In contrast, this work does not provide
applications with direct access and control over BNVM data structures; instead,
it provides a key-value interface for applications.

More projects focus on building scalable storage arrays with block-based
non-volatile memory~\cite{Marmol:2014,caulfield:micro10,lim:sosp11,wu:atc15,debnath:vldb10}. We are focusing
on byte-addressable non-volatile memory on the memory bus. Other projects focus
on building full file systems for non-volatile
memory~\cite{wu:asplos94,xu:fast16,xu:sosp17,dulloor:eurosys14}, which is orthogonal to this work because file
systems fullfill different goals than key-value stores.

\paragraph{Hash Tables as BNVM Indexing Structures}

Many have studied the resurgence of hashing as a non-volatile-storage-based
indexing structure due to the random-access,
non-block-oriented nature of the storage technologies. Debnath \etal discuss
improving hash tables for BNVM via several techniques that primarily reduce the
cascading writes effects of cuckoo hashing while maintaining low cache
pressure~\cite{Debnath:2016ht}.
However, they do not focus on making each operation on the table transactionally
consistent nor on reducing the size of the operations which must be protected.
In contrast, while the hash table design in this work could be improved with
some of the same techniques to reduce cascading writes, it focuses on making
each sub-operation serializable and composing them to never leave the table in
an inconsistent state.

\paragraph{Cuckoo Hashing}

There are many improvements to the base cuckoo hashing
algorithm~\cite{Pagh:2004} that can be made, including improving
concurrency~\cite{Li:2014ch}, generalizing cuckoo hashing to d-way
hashing~\cite{Fotakis:hashing}, and cuckoo hashing with CLOCK-style
eviction~\cite{Fan:2013}. These improvements, among the vast numbers of existing
improvements to cuckoo hashing, are not orthogonal to this work, and many could
be used in conjunction with the optimizations and designs presented here.

\paragraph{KV Stores and DBMSs}

There have been numerous projects looking at building key-value stores or
database management systems for non-volatile memories. For example, Arulraj
\etal discuss the implications of BNVM for a DBMS~\cite{Arulraj:2017}. While
many of the lessons there are relevant for key-value stores, many of the design
requirements and implementation details are due to the additional requirements
of a DBMS over a key-value store, and they do not take into account
transactional memory limitations.


Echo is another key-value store designed for non-volatile main
memories~\cite{echo}, however it is also designed with DRAM caching and an
explicit commit operation in mind. In constrast, NVKV persists automatically on
every operation without an interposition of DRAM. A similar problem can be
found in Write-behind logging~\cite{Arulraj:2016wbl}, which uses a novel approach to do logging on
transactions in non-volatile memory. It similarly limits itself to copying
between DRAM and BNVM, as does NVMcached~\cite{Wu:2016}, neither of which
take transactional memory into account.



\section{Future Work}

Future work can be broken down into three broad categories:
\begin{enumerate}
\item \textbf{Expanding}: The functionality of NVKV is extremely limited right
now. There are two major features that are missing that would both be
challenging to implement, but also extremely good candidates for research:
concurrency and larger transactions.

NVKV is currently single-threaded and does not support concurrent operations.
Doing so would be somewhat challenging, since most persistency-safe
transactional memory models do not take concurrent operations into account. The
simple solution is to use locks, resetting them should the power fail. However,
this can limit the multithreaded performance of the application. It would be
interesting to investigate how more fine-grained concurrency would be possible
while remaining consistent across power failures.

Next, NVKV uses transactional memory to ensure its consistency with each
operation, but it does not
provide support for multi-operation transactions like \bdb does. While support
for such transactions could be implemented in a similar manner, it would be
interesting to determine the most viable way to implement multi-operation
transactions.

\item \textbf{Optimizing}: A significant performance loss for insert is in the
data copy-in phase. Each written word of the data is separately persisted before
the next one is written. Language support for a
\textit{persists-with} relationship\footnote{This was my class project in
Programming Languages---defining the math for this to work. It is possible and
can be well defined.} where you could specify that if a particular write $x$
is persisted, then a set of writes $w_0, w_1, ..., w_n$ must also be
persisted, would allow a set of optimizations and improvements that are
extremely easy to use compared to explicit flushing.
This effectively enforces a visible ordering on persistent writes. It can be
implemented with much higher efficiency than explicit cache-line flushing, and
would significantly improve the data copy-in phase.

Another optimization would be for lookup. If each value were tagged with the
``level'' of the hash table that it was locatable by, and this tag were updated
each time it moved, then the fsck tool could also optimize by calculating the
lowest level required for lookup to check before declaring the item not present.
Alternatively, the fsck tool (which is allowed to optimize the database) could
do the reinsert of each item in the table, as long as it first made a backup.

\item \textbf{Exploring}: Finally, the Cuckoo hashing variant presented here has
merit for further exploration and research. An obvious path would be to explore
how other variants of Cuckoo hashing (which, for example, increase the load factor that
the table can tolerate) can interoperate with the no-reinsertion while rehashing
scheme. Finally, a more detailed analysis of this scheme should be done.

\end{enumerate}

\section{Conclusion}

Non-volatile memory opens up the possibility of a new generation of applications
with new, fundamentally different designs compared to applications designed for
a separated storage heirarchy. Key-value stores need to be similarly redesigned
with the new consistency requirements and low-latency access in mind. NVKV
attempts to tackle these challenges through the use of \textit{direct} access to
BNVM and transactional memory to ensure consistency. These two requirements
resulted in a system with small transacation sizes, always-consistent
sub-operations, and bounded steps for insert.

It is possible that transactional memory will appear differently than it is
presented in this paper. However, we expect that the pessimistic nature of our
transactional memory use will allow it to be applicable to a wide variety of
transactional memory implementations. We found that the transactions on memory
required by insert were small, and that other memory operations could be
implemented without transactions. The transaction sizes were both improved and
worsened when compiling with size optimizations, and each transaction was not
optimal in terms of instruction and memory access count, pointing to the need
for language support for transactions. We found that a significant improvement
to the design, code complexity, and performance of the system could be
achieved by using the non-reinsert rehashing scheme described above. Cuckoo
hashing fits well with this scheme, resulting in unexpected positive effects
when using it. Finally, the overall performance of NVKV was significantly
improved over \bdb, likely due to implementing consistency mechanisms taylored
for the technology rather than grafting old mechanisms onto a new storage heirarchy.


Certainly, NVKV has significant design faults and limitations; however, it
clearly demonstrates the need to redesign applications for the underlying
storage technology. This is not a surprise; we have seen this numerous times as
technologies become cheaper or more readily available. However, the longer we
wait, the harder it will be to adopt new designs for applications as the
stop-gap measures that are introduced (such as BNVM presented as block-devices)
become entrenched. We must pave a new way forward with exciting new ideas
to make best use of these technologies and not hold ourselves to designs that
will be relics of the past.









\bibliographystyle{plain}
\bibliography{bib/csrg,main,pcm_hardware}

\end{document}

