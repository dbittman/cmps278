\section{Introduction}

Byte-addressable non-volatile memory (BNVM) promises to fundamentally change
how applications access persistent
storage~\cite{lee_architecting_2009,fox:2001feram,sttram,wong2010phase,intel3dxpoint},
especially when the memory hierarchy changes to include BNVM along-side DRAM.
Changes are expected across the system stack~\cite{condit:sosp09}, from
processors introducing new feature sets, to operating systems providing new I/O
models, to applications being redesigned around low-latency persistent memory. Of
particular note is the additional \textit{power} given to middleware
applications---the operating system will provide more direct access to
persistent storage to middleware applications and
libraries~\cite{bittman-ssrctr-17-01}, allowing these applications more control
over their storage techniques and access without operating system interposition.
This has been a long debated topic, and with the advent of BNVM the debate is
coming closer to being settled in favor of applications getting more control and
power.

Of course, with greater power comes greater responsibility, and this is even
more true with BNVM. Data consistency issues threaten most applications, and the
problems of consistency are only magnified with the fine-grained writes of
byte-addressable storage. Whereas writes to DRAM would disappear when power is
cycled, corrupted data structures in BNVM persist across power cycles, meaning
applications must expend additional effort to prevent such corruption that was
not required when the storage hierarchy was separated\footnote{We could
rely on block-oriented storage to ease this.}. An additional
difficulty is energy use, where prior systems could both buffer writes in DRAM
(which had power usage characteristics that were largely independent of write
bandwidth) and coalesce them to persistent storage. With BNVM, however, the
power scales with write bandwidth at a much higher rate, meaning writes must be
minimized to minimize power. Minimizing writes has the additional benefit of
reducing wear on the memory cells, which is important since many candidate
technologies have significantly lower write endurance than DRAM.

To explore these problems, we have developed NVKV, a non-volatile key-value
store that provides backwards compatibility with the \bdb programming interface.
The prototype implements insert and lookup, although delete would also be trivial
to implement. It organizes and stores data in such a way so that it is easy to
perform all operations using simple and small memory transactions provided by
future software or hardware transactional memory. To support such a design, it
indexes data using a variant of Cuckoo hashing that does not require moving
items when rehashing them. We found that this design allowed the transactions
required to implement insert to be small (fewer than 5 memory access), and that
it was relatively easy to ensure the database could never be corrupted by
unexpected power failures. Furthermore, we found the performance to be
significantly improved over \bdb, likely because our system was designed for
such a memory hierarchy, where \bdb was not. Finally, the Cuckoo hashing variant
we designed resulted in significantly reduced data movement, which is important
for reducing power and wear on BNVM.

The main contributions of this work are:

\begin{enumerate}

\item We motivate the need to reevaluate the design of a key-value store for
BVNM.
\item We build a key-value store, NVKV, which uses transactional
memory to implement updates, and show that it has significantly improved
performance over \bdb.
\item We show that a simple mechanism to minimize writes (simply not reinserting
after expanding the hash table) is viable and does not significantly hurt
overall performance (and in fact smooths latency spikes caused by expanding the
hash table).
\end{enumerate}

