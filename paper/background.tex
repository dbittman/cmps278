\section{Background}

Non-volatile memory technologies, including phase change memory
(PCM)~\cite{lee_architecting_2009,wong2010phase}, Ferroelectric RAM
(FeRAM)~\cite{fox:2001feram}, and spin-torque transfer RAM (STT-RAM)~\cite{sttram},
among others, promise to fundamentally change the design of our
devices, operating systems, and applications. Although the technologies are
starting to make their way into consumer devices~\cite{intel3dxpoint}, their
full potential will be seen when they replace or exist alongside DRAM on the
memory bus (as byte-addressable non-volatile memory, BNVM).
Such a memory hierarchy will allow the processor, and therefore
applications, to access persistent storage with normal load and store
instructions, bypassing the high-latency I/O operations of the operating system.

The advent of these technologies means that we must build new applications
explicitly designed for them. Whereas existing key-value stores would work when
given access to a file in a file system on BNVM, the result would be sub-optimal
because it would still have the operating system interposing on access to
persistent storage, which is unnecessary for BNVM. Furthermore, the access would
be block oriented, which is irrelevant for these technologies. Finally, the
consistency mechanisms enforced by file systems would be overkill compared to
the more light-weight consistency that could be achieved by a key-value store
designed for BNVM. Together, these all act as additional overhead for key-value
stores that could be avoided by properly redesigning and reevaluating the needs
of such a system. That is a goal of NVKV; understanding the underlying
technology and designing for it.

While this is not new, the particular system model NVKV is built for is. We are
entering an era where we may see a range of persistent memory on the memory bus
from no persistent memory to \textit{only} persistent memory. While some exising
key-value stores are built for persistent memory, they often assume the
existence of DRAM~\cite{echo,Arulraj:2016wbl}. However, IoT devices may soon see
their use of DRAM diminish as more power-friendly technologies come
out~\cite{Jayakumar2014powering}. In fact, from a survey on
papers~\cite{dhiman_pdram:_2009,lee_architecting_2009,xiangyu_dong_nvsim:_2012,qureshi_scalable_2009,Chen_rethinkingdatabase,bedeschi_8mb_2004} discussing
energy consumption of PCM and DRAM, we can see that the energy cost per bit to
write to PCM is 50 times more expensive, but
the idle cost of DRAM is a \textit{billion} times the write cost per bit. This
means that the idle cost of DRAM can only be surpassed by PCM when writing a
billion bits per second. While this is large, it is not unreachable---systems
must be careful to avoid excessive copies and rewrites when using PCM, something 
existing systems do not always optimize for. Finally, since PCM is
extremely energy efficient in read-mostly workloads, it will likely find its
place in IoT devices without DRAM. Thus, key-value stores must be designed for
this all-BNVM model.

\paragraph{Transactional Memory}

Many BNVM systems focus on consistency of data structures due to the
persistent nature of BNVM. When applications store persistent data in
persistent memory, they expect the data to remain consistent with respect to a
set of invariants (e.g. a linked list is valid, etc). However, with the fine
granularity of writes to BNVM and the ability for power failures to strike at
any time, ensuring consistency on BNVM is an important challenge that must be
addressed in new ways.

There are many ways to provide applications with methods for ensuring
consistency, ranging from persisting during shutdown with
battery-backups~\cite{narayanan:asplos12} to write-through caching and careful
ordering with atomics~\cite{bhandari2012implications}, to explicit flushing and
fencing~\cite{condit:sosp09}. While these mechanisms are usable, they range from
being non-portable to being incredibly inefficient. Only explicit flushing is
close to reasonable in terms of both usability and performance, but it has the
same problems as atomic memory accesses without programming language support.
Only with proper language support can persistent memory be both high performance
and easily usable.

The exact form of this language support is still debated, partly because it will
depend on the type of hardware support we see for persistent memory programming.
Hardware support could range from persist barriers to support for transactional
memory. However, existing hardware transactional memory (e.g. in Intel
processors) is deprecated due to bugs. Instead, we are left with enough support
for software transactional memory, which has similar problems as other
approches, namely performance and usability~\cite{stm}. However, hardware
transactional memory \textit{could} be a
reality~\cite{kolli:asplos16,lu:tos16,wang:cal15}---a reality this paper
assumes. Thus, we are focusing on developing a key-value store built for BNVM
with hardware (or software) transactional memory support to ensure consistency.
As an optimization, we additionally make use of weaker consistency mechanisms,
such as fencing and flushing, when transactions would be too heavy-weight for
the updates we are protecting.

