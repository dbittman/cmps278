\section{Related Work}

\paragraph{BNVM Systems}

There has been a lot of recent work on providing clean and safe abstractions for
non-volatile memory, including data
structures~\cite{coburn:asplos11,hu:atc17,Yang:2015,Venkataraman:2011} and programming
models~\cite{ren:micro15,volos:asplos11,condit:sosp09,guerra:atc12,Narayanan:2012}.
Most of these provide a lower-level interface for applications
to be built atop non-volatile memory and focus on consistency and durability. In contrast, this work does not provide
applications with direct access and control over BNVM data structures; instead,
it provides a key-value interface for applications.

More projects focus on building scalable storage arrays with block-based
non-volatile memory~\cite{Marmol:2014,caulfield:micro10,lim:sosp11,wu:atc15,debnath:vldb10}. We are focusing
on byte-addressable non-volatile memory on the memory bus. Other projects focus
on building full file systems for non-volatile
memory~\cite{wu:asplos94,xu:fast16,xu:sosp17,dulloor:eurosys14}, which is orthogonal to this work because file
systems fullfill different goals than key-value stores.

\paragraph{Hash Tables as BNVM Indexing Structures}

Many have studied the resurgence of hashing as a non-volatile-storage-based
indexing structure due to the random-access,
non-block-oriented nature of the storage technologies. Debnath \etal discuss
improving hash tables for BNVM via several techniques that primarily reduce the
cascading writes effects of cuckoo hashing while maintaining low cache
pressure~\cite{Debnath:2016ht}.
However, they do not focus on making each operation on the table transactionally
consistent nor on reducing the size of the operations which must be protected.
In contrast, while the hash table design in this work could be improved with
some of the same techniques to reduce cascading writes, it focuses on making
each sub-operation serializable and composing them to never leave the table in
an inconsistent state.

\paragraph{Cuckoo Hashing}

There are many improvements to the base cuckoo hashing
algorithm~\cite{Pagh:2004} that can be made, including improving
concurrency~\cite{Li:2014ch}, generalizing cuckoo hashing to d-way
hashing~\cite{Fotakis:hashing}, and cuckoo hashing with CLOCK-style
eviction~\cite{Fan:2013}. These improvements, among the vast numbers of existing
improvements to cuckoo hashing, are not orthogonal to this work, and many could
be used in conjunction with the optimizations and designs presented here.

\paragraph{KV Stores and DBMSs}

There have been numerous projects looking at building key-value stores or
database management systems for non-volatile memories. For example, Arulraj
\etal discuss the implications of BNVM for a DBMS~\cite{Arulraj:2017}. While
many of the lessons there are relevant for key-value stores, many of the design
requirements and implementation details are due to the additional requirements
of a DBMS over a key-value store, and they do not take into account
transactional memory limitations.


Echo is another key-value store designed for non-volatile main
memories~\cite{echo}, however it is also designed with DRAM caching and an
explicit commit operation in mind. In constrast, NVKV persists automatically on
every operation without an interposition of DRAM. A similar problem can be
found in Write-behind logging~\cite{Arulraj:2016wbl}, which uses a novel approach to do logging on
transactions in non-volatile memory. It similarly limits itself to copying
between DRAM and BNVM, as does NVMcached~\cite{Wu:2016}, neither of which
take transactional memory into account.

